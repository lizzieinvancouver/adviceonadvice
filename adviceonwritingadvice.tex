\documentclass[11pt,letter]{article}
\usepackage[top=1.00in, bottom=1.0in, left=1.1in, right=1.1in]{geometry}
\renewcommand{\baselinestretch}{1.1}
\usepackage{graphicx}
\usepackage{natbib}
\usepackage{amsmath}
\usepackage{parskip}
\usepackage{hyperref}

\def\labelitemi{--}
\parindent=0pt

\begin{document}
\bibliographystyle{/Users/Lizzie/Documents/EndnoteRelated/Bibtex/styles/besjournals}

\title{Advice on Writing Advice}
\author{Andrew Gelman  \& E. M. Wolkovich}
\date{Last updated: \today}
\maketitle

\begin{abstract}
Writing rarely comes naturally. It typically requires lots of practice to learn to write well.  Unsurprisingly then, there is a lot of writing advice. Here we offer some advice to students writing research articles on how to make use of these resources. 
\end{abstract}

\section{Main text} % Short version

Lots of things that seem like they should be easy and are actually hard. This includes playing music, many sports (e.g. basketball, snooker, cross country skiing) and other things we may have tried as kids (sewing, woodworking) that appear easy and may be easy to start doing, but are hard to do very well. Writing is one of these things, and perhaps a particularly painful one as we often think that writing should be easy. 

It is easy enough to write a text or email and convey what you’re trying to say, but formal writing is much more difficult. Formal writing can feel especially challenging because: (1) it's non-algorithmic---there's no repeatable recipe to follow the way there is for many lab assays or casseroles, (2) we generally don't know what we want to say ahead of time, and (3) without the natural flow of a back-and-forth conversation it's hard for writers tell what the reader needs. All of this can make writing feel especially frustrating. The result is that many academics, whose profession is about the development and transmission of new knowledge, write poorly. 

We argue that most academic writing is bad for the same reason that most writing is bad: because writing is hard.  It’s difficult to write clearly, it takes effort and it takes practice, and, on top of all that, many people don’t see any continuous path of effort that would take any particular piece of writing from bad to good.  You can immediately tell that a soup tastes bad; that doesn’t mean you know what should be added or subtracted from the recipe to turn it into the delicious meal that you would like.  In that way, writing is like drawing:  we can know what we’re trying to convey, recognize that our depiction is terrible, but have no idea what direction to go in to improve it.

The combination of the central importance of writing to conveying our brilliant new knowledge and the complexity of making bad writing better means the market for advice on how to write better is vast. We know because we have read a chunk of it. We’re not writing teachers, but we write for our professions. As academics ourselves who want to write better, we've each read a lot of writing advice, from books to articles to blogs and in between. We know the amount of advice, and its sometimes contradictory nature, can feel frustrating itself---adding frustration to an already frustrating task. Thus, here we provide short advice on how to benefit from writing advice. 

\emph{Our advice on advice}

If you want to write well you generally need some faith in the reality that it is hard and thus some dedication to the effort it will take to get better at something that is hard. Reading writing advice and trying to implement useful advice is one major path to better writing, alongside two often mentioned pieces of writing advice: write regularly, and read. 

Once you start reading writing advice though you may stumble across the following problems, which we suggest---with some additional explanation---you muster your way through:

\begin{enumerate}
\item {\bf You dislike the advice.} This happens a lot. For example, we compared writing books we like and found we often hated the ones the other liked. The advice often was not very different---though sometimes it was---but the tone or format annoyed one of us, while the other was enthralled or at least saw its utility. 

The reality is that you don't have to like writing advice to gain from reading about it. If you read enough about writing, you'll find pieces that stay with you and help you write better. And, as you wade through the parts you don't like, you may slowly realize some of it is advice you desperately need. If not, you might at least learn where old rules you have heard and disagree with come from (for example, why your postdoc advisor told you to never start a sentence with `however'). % Strunk \& White 

Writing advice is usually fairly well written. So, even if you feel horribly at odds with the advice, you're reading some decent writing. This generally makes you write better. 
\item {\bf You believe the advice does not apply to you.} Some of it likely doesn't, but much of it does. Good writing is non-algorithmic, but sentences, paragraphs and papers that make sense to people often follow certain rules. You likely appreciate writing that follows many of these rules, such as giving readers information in an order they can digest or putting groups of ideas together. 

Some writers enjoyably break these rules. However, much like Picasso, they are often experts at following the rules before they try breaking them. We suggest you follow this approach as well. Be uniquely gifted at excellent rule-following writing before you break out of the box. If you do start breaking out of the box, solicit and take feedback on whether it's working. 
\end{enumerate}

Now that we're past two common problems you may run into, here's some final advice on implementing all the advice you're reading.
\begin{enumerate}
\item {\bf If you see advice repeatedly in recent books on how to write, try it.} This follows from our previous advice, but is a little more nuanced. 

Writing books comes in all sorts of shapes and sizes, from various historical decades, cultural norms and distinct viewpoints. You might not like all the advice. You might disagree with it vehemently. But if you see the same point made across many different sources of writing advice, you should try to follow it. Remember you're writing for your career so you can always try advice for a month, a year---or two---then decide if it is working. But if you don't try it, then you may be tacitly committing to stick with your poor writing. 
\item {\bf Remember, getting better at writing takes more than just reading advice about writing.} Read writing advice! Read,  `my beautiful intellectuals,' read (adapted from \href{https://www.youtube.com/watch?v=yoEezZD71sc}{Tim Minchin UWA address}).  But don't forget that to get better at writing you have to write. % But don't forget that like any sport or other skill, practice is required. You have to write to get better at writing. "Empathy is intuitive but is also something you can work on."
\end{enumerate}


\emph{Recommendations}

If reading this has gotten you excited by the idea of reading some writing advice, here's a short list to consider. 
\begin{enumerate}
\item \emph{On Writing Well: The Classic Guide to Writing Non-fiction} by William Zinsser is a beautiful book, with all sorts of fantastic and depressingly simple advice, written so well that you can re-read it many times. 
\item Strunk \& White’s \emph{The Elements of Style} was written separately in time by Will Strunk and his student E. B. White (of \emph{Charlotte’s Web} fame). Strunk wrote the original commands, which White wrote up later, with some adjusting and added a section, `An Approach to Style,’ at the end. The book then is a mixture of types of advice: grammatical advice interwoven with ideas about organization and style. While parts definitely feel anachronistic, it is filled with good advice and wonderfully short. The writing by White, in particular, shows a man in possession of great writing skill, admitting all the traumas and tribulations of writing alongside advice of how to succeed. 
\item Williams and Bizup's \emph{Style: Lessons in Clarity \& Grace}---and the much smaller `Basics' edition---lays out how to write clearly and directly with examples of each common problem and how they are fixed. 
\item \emph{How to Write a Lot: A Practical Guide to Productive Academic Writing} by Paul Silvia was ideal for E.M.W. when she was procrastinating about writing her PhD by reading about writing because it lays out all the rules about writing in short order (which are: to write a lot, you need to sit down and write, regularly). 
\item A couple more of Andrew's top picks
\end{enumerate}

This an extremely short list. It's missing a number of excellent books, as well as entire types of writing books you should consider reading. For example, books and articles on creativity or engaging your audience, books specific to your subject that cover the intricacies of the full process (in ecology, this would include \emph{Stephen Heard’s The Scientist’s Guide to Writing}, which says a bunch of the stuff you’re supposed to pick up from the ether: how to make a figure, write in the tense expected in different sections of a research article, pick a journal), or books dedicated to the love of punctuation. Ideally you'll be reading a lot and cover these genres as you go. We also provide a longer list of books in our appendix. 

We cannot resist the urge to give our top list of writing advice...
\begin{enumerate}
\item Write for your reader and be kind to them. (Similar: Show respect for your reader.) Or, from Strunk \& White: have "… deep sympathy for the reader. Will [Strunk] felt that the reader in serious trouble most of the time, floundering in a swamp, and that it was the duty of anyone attempting to write English to drain this swamp quickly and get the reader up on dry ground." Orwell: Totalitarians write poorly because they want to obfuscate.
\item Focus on immediate and long-term goals and skip the intermediate goals (like getting your article accepted, published etc.). This also connects to saying why what you did is important. 
\item Make sure you have the much-needed gap. 
\item Live to write another day (you don’t have to get everything you know or found in this one article/book/paper).
\item Accept that you’ll revise. 
\item Learn the basic rules before you break them: Strunk, “Unless he is certain of doing as well, he will probably do best to follow his rules.” This is why I like Williams’ Style. It’s dull as all tomorrow, but it’s got the damn rules all written down with lots of examples of how they work. 
\item Murder your darlings. This advice is often given (and attributed), it generally means – if you’re in love with some turn of phrase or such that you have written, take it out. We all fall in love with things we write and hold tight to them; when you get feedback to delete something you love, remember this advice. But it also relates, for me, to learning the rules before you break them…
\item Outline. I like how Strunk \& White said this: (Ch II) “ In most cases, planning must be a deliberate prelude to writing.” Or from White’s ending chapter: Work from a suitable design.
\item Think of the paragraph as the unit of writing. 
\item Omit needless words
\end{enumerate}

\section{Appendix: A longer list of books/articles}

\begin{enumerate}
\item Understanding Comics: The Invisible Art by Scott McCloud 
\item  Stuff by Robert Boice is also good, such as the writing section of \emph{Advice for New Faculty Members}, it shows some graphs in case you still want to believe spontaneous writers write better.
\item Made to Stick: Why Some Ideas Survive and Others Die (Heath \& Heath) is not written as well as I would like, but all the points are valid and not put together so well anywhere else (that I am aware of).
\item The article that slams Strunk and White (??)
\item Loehle's (article) classic on creativity in science compares your ideas to a zoo of animals, in a very good way. \emph{A guide to increased creativity in research---inspiration or perspiration?}
\item Pinker
\item Orwell
\item Fowler
\tem Becker
\item Higgins
\item Virginia Tufte 
\item Josh Schimel's Writing Science: How to Write Papers That Get Cited and Proposals That Get Funded
\item Stephen Heard’s The Scientist’s Guide to Writing (\url{https://press.princeton.edu/books/paperback/9780691170220/the-scientists-guide-to-writing}), which I like as it says a bunch of the stuff you’re supposed to pick up from the ether: how to make a figure, write in the tense expected in different sections of a research article, pick a journal, how to get your paper rejected and move on and upward with the arts … on and on. 
\emph{Chicago Manual of Style} for arguments about whether you can capitalize a word after a semi-colon, plus they have a \emph{Manual for Writers of Research Papers, Theses and Dissertations} that is handy.
\item My own advice for writing research articles:  \url{https://statmodeling.stat.columbia.edu/2014/01/14/advice-writing-research-articles/}
\item Basbøll 
\item Beverly Cleary 
\item If, but, therefore book. 
\end{enumerate}



\section{Appendix: Incomplete longer version of writing advice}
\begin{enumerate}
\item "Keep it simple while telling a coherent, compelling story" (McCarthy)
\item Read writing advice! People who write it are often good writers so it’s good reading and thinking/reading/working on writing better likely will improve your writing.
\item Read more. To be a good writer you need to read good writing and develop an ear for what sounds good, and maybe eventually you’ll notice why it works (if you read a lot of writing advice especially). 
\item Take advice. Even if you love how to write, try advice you’re given, especially advice you have a feeling may apply to you (and/or if you see it often). Getting too attached to your style can prevent this, so remember to let go a little. Try it for a while, see how it feels, see how readers respond—you can always go back to your old style, but you cannot make your writing better if you never change it. 
\item Remember your reader when receiving advice. Writing is less about you and more about your reader, so when people read your writing and give you advice, you’re getting info from a most valuable source—a reader: appreciate this and take it under real consideration, especially when your readers may have very different perspectives from you. For example, non-native English speakers often tell me things about English words that are not true for native English or American speakers, (for example, ‘alter’ has a negative connotation) but I often find they are true for second-language English speakers and for some of my research the main audience is second- language speakers so I try to make the changes. 
\item Write the talk first, then write the paper from the talk (from my colleague Ben Cook): often it becomes clear in a talk what’s missing or mis-organized, so sorting this out first can help you write a better paper later. 
\end{enumerate}
 % “Decide on your paper’s theme and two or three points you want every reader to remember” (McCarthy); this might mean you have to make some other points in other papers, or just let some go. It’s better to make a couple points the reader can take in, then make so many that none actually stick. “If something isn’t needed to help the reader understand the main theme, omit it.”
% “Try not to think about the paper until the reviewers and editors come back with their own perspectives.” (McCarthy)
% “Don’t slow the reader down.” (McCarthy)

% \section{Next steps ...}


\section{Miscellaneous}
Berger 1990
\begin{quote}
According to Nordic myth, the modem world begins when Odin slays Ymir, the Ice Giant. Ymir's offspring drown in his blood, but two survive and start a new race of frost giants. Lodged in Utgard, they pose a constant threat. Only Odin's son Thor, brandishing Mjollnir, the magic hammer, keeps them in check. The pronounced and abrupt changes in climate during the Glacial-Holocene transition suggest that the luck of battle switched sides frequently before Odin and Thor won over Ymir and his kin.
\end{quote}



\end{document}