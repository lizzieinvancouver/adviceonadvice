\documentclass[11pt,letter]{article}
\usepackage[top=1.00in, bottom=1.0in, left=1.1in, right=1.1in]{geometry}
\renewcommand{\baselinestretch}{1.1}
\usepackage{graphicx}
\usepackage{natbib}
\usepackage{amsmath}
\usepackage{parskip}
\usepackage{hyperref}

\def\labelitemi{--}
\parindent=0pt

\begin{document}
\bibliographystyle{/Users/Lizzie/Documents/EndnoteRelated/Bibtex/styles/besjournals}

\title{Advice on Writing Advice}
\author{Andrew Gelman  and E. M. Wolkovich}
\date{Last updated: \today}
\maketitle

\begin{abstract}
Writing rarely comes naturally. It typically requires lots of practice to learn to write well.  Unsurprisingly, then, there is a lot of writing advice out there. Here we offer some advice to students writing research articles on how to make use of these resources. 
\end{abstract}

\section{Short version}

Lots of things seem like they should be easy and are actually hard. This includes many sports (basketball, snooker, cross country skiing) and other things we may have tried as kids (sewing, woodworking) that appear easy and easy to start doing, but are hard to do very well. Writing is one of these things, and perhaps a particularly painful one as we often think that writing should be easy. 

It is easy enough to write a text or email and convey what you’re trying to say, but formal writing is much more difficult. Formal writing can feel especially challenging because: (1) it's non-algorithmic---there's no repeatably recipe to follow the way there is for many lab assays, (2) we generally don't know what we want to say ahead of time, and (3) without the natural flow of a back-and-forth conversation it's hard for writers tell what the reader needs. All of this can make writing feel especially frustrating, not just because writing is hard, but also because it feels that it should be easy. The result is that much academics, who are in a profession is about the development and transmission of new knowledge, write poorly. 

We argue that most academic writing is bad for the same reason that most writing is bad: because writing is hard.  It’s difficult to write clearly, it takes effort and it takes practice, and, on top of all that, many people don’t see any continuous path of effort that would take any particular piece of writing from bad to good.  You can immediately tell that a soup tastes bad; that doesn’t mean you know what should be added or subtracted from the recipe to turn it into the delicious meal that you would like.  In that way, writing is like drawing:  we can know what we’re trying to convey, recognize that our depiction is terrible, but have no idea what direction to go to improve it.

The combination of the central importance of writing to conveying our brilliant new knowledge and the complexity of making bad writing better means the market for advice on how to write better is wide. We know, as we have read a chunk of it. We’re not writing teachers, but we write for our professions. As academics ourselves who want to write better, we've each read a lot of writing advice, from books to articles to blogs and in between. We know the amount of advice, and its sometimes contradictory nature, can feel frustrating itself---adding frustration to an already frustrating task. Thus, here we provide short advice on how to benefit from writing advice. 

\emph{Our advice on advice}

If you want to write well you generally need some faith in the reality that it is hard and thus some dedication to the effort it will take to get better at something that is hard. Reading writing advice and trying to implement is one major path to better writing, alongside two often mentioned pieces of writing advice: write regularly, and read. 

Once you start reading writing advice though you may stumble across the following problems, which we suggest---with some additional explanation---you muster your way through:

\begin{enumerate}
\item {\bf You dislike the advice.} This happens a lot. For example, we compared writing books we like and found we often hated the ones the other liked. The advice often was necessarily different---though sometimes it was---but the tone or format annoyed one of us, while the other was enthralled or at least saw its utility. 

The reality is that you don't have to like writing advice to gain from reading about it. If you read enough about writing, you'll find pieces that stay with you and help you write better. And, as you wade through the parts you don't like, you may slowly realize some of it is advice you desperately need. If not, you might least learn where old rules you have heard and disagree with come from (for example, why your postdoc advisor told you to never start a sentence with `however'---Strunk \& White). 

Writing advice is usually fairly well written. So, even if feel horribly at odds with the advice, you're reading some decent writing. This generally makes you write better. 
\item {\bf You believe the advice does not apply to you.} Some of it likely doesn't, but much of it does. Good writing is non-algorithmic, but sentences, paragraphs and papers that make sense to people often follow certain rules. You likely appreciate writing that follows many of these rules, such as giving readers information in an order they can digest or putting groups of ideas together. 

Some writers enjoyably break these rules. However, much like Picasso, they are often experts at following them before they ever try breaking them. We suggest you follow this approach as well. Be uniquely gifted at excellent rule-following writing before you break out of the box. If you do start breaking out of the box, solicit and take feedback on whether it's working. 
\item {\bf If you see advice repeatedly in recent books on how to write, try it.} This follows from our previous advice, but is a little more nuanced. 

Writing books comes in all sorts of shapes and sizes, from various historical decades, cultural norms and distinct viewpoints. You might might not like all the advice. You might disagree with it vehemently. But if you see the same point made across many different sources of writing advice, you should try to follow it. Remember you're writing for your career so you can always try advice for a month, a year---or two---then decide if it is working. But if you don't try it, then you may be tacitly committing to stick with your poor writing. 
\item {\bf Remember, getting better at writing takes more than just reading advice about writing.} Read writing advice! Read,  `my beautiful intellectuals,' read.  But don't forget that to get better at writing you have to write.  (\href{https://www.youtube.com/watch?v=yoEezZD71sc}{Tim Minchin UWA graduation address} % But don't forget that like any sport or other skill, practice is required. You have to write to get better at writing. "Empathy is intuitive but is also something you can work on."
\end{enumerate}


\emph{Recommendations}

If reading our advice on writing advice has gotten you excited by the idea of reading writing advice, here's some to consider. 

We cannot resist the urge to give our top list of writing advice... 


\section{Incomplete longer version}


\section{Next steps ...}
\begin{enumerate}
\item The word doc has a lot of stuff. I suggest leave it all there and start here anew...
\item Come up with a quick outline here.
\item Write a short form of the paper and send to Andrew. 
\end{enumerate}

\section{Miscellaneous}
Berger 1990
\begin{quote}
According to Nordic myth, the modem world begins when Odin slays Ymir, the Ice Giant. Ymir's offspring drown in his blood, but two survive and start a new race of frost giants. Lodged in Utgard, they pose a constant threat. Only Odin's son Thor, brandishing Mjollnir, the magic hammer, keeps them in check. The pro- nounced and abrupt changes in climate during the Glacial-Holocene transition suggest that the luck of battle switched sides frequently before Odin and Thor won over Ymir and his kina.
\end{quote}

\end{document}